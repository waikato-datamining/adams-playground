% Copyright (c) 2024 by the University of Waikato, Hamilton, NZ. 
% This work is made available under the terms of the 
% Creative Commons Attribution-ShareAlike 4.0 license,
% http://creativecommons.org/licenses/by-sa/4.0/.

\documentclass[a4paper]{book}

\usepackage{wrapfig}
\usepackage{graphicx}
\usepackage{hyperref}
\usepackage{multirow}
\usepackage{scalefnt}
\usepackage{tikz}
\usepackage{varwidth}

% watermark -- for draft stage
\usepackage[firstpage]{draftwatermark}
\SetWatermarkLightness{0.9}
\SetWatermarkScale{5}

\input{latex_extensions}

\title{
  \textbf{ADAMS} \\
  {\Large \textbf{A}dvanced \textbf{D}ata mining \textbf{A}nd \textbf{M}achine
  learning \textbf{S}ystem} \\
  {\Large Module: adams-djl} \\
  \vspace{1cm}
  \includegraphics[width=2cm]{images/djl-module.png} \\
}
\author{
  Peter Reutemann
}

\setcounter{secnumdepth}{3}
\setcounter{tocdepth}{3}

\begin{document}

\begin{titlepage}
\maketitle

\thispagestyle{empty}
\center
\begin{table}[b]
	\begin{tabular}{c l l}
		\parbox[c][2cm]{2cm}{\copyright 2024} &
		\parbox[c][2cm]{5cm}{\includegraphics[width=5cm]{images/coat_of_arms.pdf}} \\
	\end{tabular}
	\includegraphics[width=12cm]{images/cc.png} \\
\end{table}

\end{titlepage}

\tableofcontents
%\listoffigures
%\listoftables


%%%%%%%%%%%%%%%%%%%%%%%%%%%%%%%%%%%
\chapter{Introduction}
Deep Java Library (DJL) is an open-source, high-level, engine-agnostic Java framework for deep learning.
DJL is designed to be easy to get started with and simple to use for Java developers. DJL provides a native
Java development experience and functions like any other regular Java library.\footnote{\url{https://docs.djl.ai/master/index.html}{}}

\section{Cache}
Native engine libraries and models get cached here:
\begin{tight_itemize}
    \item Linux/Mac: \verb|$HOME/.djl.ai|
    \item Windows: \verb|%USERPROFILE%/.djl.ai|
\end{tight_itemize}
For more details, see the DJL documentation: \\
\url{https://docs.djl.ai/master/docs/development/cache_management.html}{}

%%%%%%%%%%%%%%%%%%%%%%%%%%%%%%%%%%%
\input{bibliography}

\end{document}
